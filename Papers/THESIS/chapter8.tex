\chapter{Conclusion and Further Work}

\section{Conclusion}
In conclusion, this paper examined the literature in regards to artificial intelligence and its use for facial sentiment analysis. Five papers were described in detail and reviewed in terms of quality consistency. It was found that neuromarketing with artificial intelligence could deem to be a very useful tool for preventing customer churn. Following the assessment of current literature, an analysis was brought forward to give justification for the project and why such a tool would be useful. As stated, only 10 percent of customer service emails receive a response and that artificial intelligence can automate this tedious task. Furthermore, the technologies required to build such a was given, which included a machine learning algorithm type and library, a large dataset and a cloud platform to host the project for public use. The design framework was layed out in Chapter 4, which describes that a number of data preparation steps need to be taken, a Tensorflow convolutional neural network needs to be implemented, training should be conducted on the FloydHub cloud service and lastly, the project and API should be deployed to a server on the Heroku PaaS. 

In Chapter 5, the implementation of the project was discussed in great detail. A number of data preparation steps were taken to cleanse and increase the size of dataset. Each image was cropped to only the facial regions and grayscaled. Data augmentation was used to create synthetic samples of the original images. A convolutional neural network was developed using the TensorFlow machine learning library. The model consisted of 10 convolution layers, 5 subsampling layers. Techniques such a shuffling and regularization were taken to prevent over fitting of the data. A web application was built with Node.JS and express to perform facial tracking of users face. 
A number of methods for testing were conducted to assess the the model and highlight some performance measures. The model was tested on unseen data and the precision, recall and accuracy were analyzed. It was seen that the model performed very well with an 84.47\% accuracy, 85.64\% precision and 84.42\% recall. However, the neutral class did not perform as well as the rest of the classes. Furthermore, usability testing was orchestrated to evaluate the performance of the model when given images for the facial cropping in the web application. 

\subsection{Further Work}
The work of this thesis project highlighted the promising features that can be added. These steps can be taken to ensure better model performance and make a more user friendly interaction with the application. The further works are as follows:
\begin{itemize}
	\item Increase the size of the training dataset by possibly using multiple datasets to provide more variance in the data. This will in turn increase the perfomance, as stated by \citeauthor{LOPES}.
	\item Perform more data preprocessing steps such as image normalization to allow the model to extract features within an image that may have high contrast lighting.
	\item Implement a tone analysis tools such as IBM Watson to give further insight into the users sentiment than just facial expressions. This will record the users voice and convert it to text, then a sentiment analysis to be performed to investigate whether the users facial expressions match the tone of their spoken words.

\end{itemize}
