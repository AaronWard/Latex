\prefacesection{Relevant Work and Critical Analysis}

\citealp{SUBHASHINI}'s work on analyzing and detecting employees emotions for organizations ties in very well with the proposed project as it involves using sentiment analysis on subject to gain an underlining understanding of their emotions that may or may not be expressed verbally. Their approach to skin tone segmentation for detecting a humans face may prove very beneficial to the proposed project. The main research goal was to achieve employee identification in conjunction with facial emotional analysis and they were successful in execution, They used a Bezier Curve on the subjects lips and eye to detect the emotion expressed by analysing the gradient of the curve. However, Although this paper proved that the project was a success, there are some inconsistencies. For example: in the related work, they do not explain how the work is related and only give titles. Secondly, the results show no code snippets or pseudocode to explain the implementation of the system. Only screen shots of the user interface are provided. Furthermore, there are more absences of proof. The paper is lacking statistics and graphs to display the accuracy or progress of the systems performance and there are some bold statements used that are not back up by citations like when it is said that "emotions were considered a forbidden topic in the working place". Despite these weak areas in the paper, a good aspect of this system is the use of real life subjects used in testing. 

The review of Deep Learning for Video Classification and Captioning by \citeauthor{Wu} provides an in depth look into the aspects of different neural networks and what are they strong and weak points. The motivation for their research is driven by their claim that video communications is growing and that their needs to be better applications for video understanding. The relevancy of this paper provides the concept of the "Two-stream" architecture. Although it is not planned to develop two convolutional neural networks (CNN), is it sought after to develop a score fusion algorithm, similar to the one mentioned in this paper. A CNN will be developed and the application will utilize a tone analyser for voice sentiment analysis and the two scores will be combined by such an algorithm to provide
the weighted sum of the two scores. As this paper is very in depths and draws good comparisons between the different techniques that can be used for video classification. Despite the quality of this paper, some aspects need improvement. There is heavy usage of words like "we" used. Also, some statements are made by \citep{Wu} that are not cited to supported their claim. This is evident when its said "As deep learning for video analysis is an emerging and vibrant feild..." (According to whom?).

Subject independent facial expression recognition with robust face detection using a convolutional neural network by \citeauthor{MATSUGU} illustrates the difficulties that may arise when performing facial recognition. They highlight the problems that may occur in terms of being able to handle variability of subject faces, and certain angles. Their approach to this problem is addressed by implementing a rule based algorithm that analysis the results given from the CNN. Furthermore, their model is designed to be segment and be trained on specific facial features instead the face as a whole. Their model proves to be a success as they score an accuracy of 97.6 percent, also they do not require a second CNN working concurrently to achieve similar results as other models have done so previous, which can be cost effective. Some similarities arise between this paper and the proposed project, they both use sentiment analysis and require the ability to handle a wide range of variability. This proves beneficial to the proposed project as it provides inspiration to used a rule based algorithm for determining emotions. Even though this paper is well written, there are some issues. In certain parts there are abbreviations to words given without the full word be given prior which can cause confusion to the reader. For example: "FP neurons". In addition to this, their model is specific for smiling faces and doesn't accommodate for other emotions, which should be at least provided in a further work section. 

 \citeauthor{arthmann}'s paper titled Neuromarketing – The Art and Science of Marketing and Neurosciences Enabled by IoT Technologies is a promising insight to the field of neuromarketing. They recognize the association of online shopping and loss of sale for retail stores, and give example of how neuro linguistic programming and IoT technologies can be used as a combination to understand their customers sentiment. Furthermore, it's heavily argued that the integration of AI and machine learning will evolve this concept to understand the thoughts of consumers and further tackle loss in productivity. This proves relevant to the proposed project as it is very similar. They both aim to gain a deeper understanding of human sentiment that may not be express verbally. Although this is possibly the most interesting paper of this review, it is severally lacking citations. Also, there are a lot of assumption brought forward with no clear indication as to how this knowledge is known. Additionally, it is also assumed that people will adopt these "always on" IoT devices and agree for their physical aspects to be used for consumer targeted marketing. 

