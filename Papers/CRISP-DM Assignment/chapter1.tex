\prefacesection{Business Understanding and Data Understanding}

\section*{Business Understanding}
This dataset provides data on subjects that with and without meningitis. It contains information such as age, gender, location, sum of health problems such as headaches, fevers and seizures. Additionally, is provides an attribute that indicate if the subject does or doesn't have meningitis, with a negative or positive value. 

\subsection*{Business Objective}

\begin{itemize}
	\item  Predict if someone is at risk of getting meningitis.
\end{itemize}

\subsection*{Data Mining objective}
\begin{itemize}
	\item The main objective is to create a model to predict the risk of a person getting meningitis 
	\item This model will use the attributes provided in the dataset such as age, gender, seizure history etc to assess the prediction accuracy.
	\item The model shall test multiple data mining algorithms to obtain a prediction.
\end{itemize}



\prefacesection{Data Understanding}

%- Describe the attributes in the dataset in terms of quality,
%information content, and usefulness
%- EDA Techniques
%- 3 charts

\subsection*{Describing Data}
In this section the allocated dataset is explained in terms of informational content, data quality and usability. The data set itself consists of attributes in relation to meningitis, a neuroligical infectious disease that can cause brain inflammation due to bacteria or viruses infecting that brain. As seen in tables below, the attributes have been segregated from numeric and nominal data. The numeric data given a description and a data type. Additionally, the mean, minimum, maximum and standard deviation values are given. Please refer to table \ref{table: numeric1} for information of the numeric data.


 \begin{longtable}{|c|>{\raggedright\arraybackslash}p{3cm}|c|c|c|c|c|}
	\hline
	\multicolumn{7}{| c |}{Numeric Attributes}\\
	\hline
	\textbf{Name} & \textbf{Description} & \textbf{Data type} & \textbf{Mean} & \textbf{Min} & \textbf{Max} & \textbf{SD}\\
	\hline
		AGE & List the age of each person & Numeric & 37.6285 & 10.0 & 84.0 & 15.3853\\
	\hline
		COLD & Number of days since last cold & Numeric & 2.6642 & 0.0 & 35.0 & 4.8273\\
	\hline
		HEADACHE & Days since last headache & Numeric & 7.1857 & 0.0 & 63.0 & 9.1278\\
	\hline
		FEVER & Days since last fevers&Numeric & 6.3428 & 0.0 & 63.0 & 8.0294\\
	\hline
		NAUSEA & Start of nausea & Numeric & 2.4857 & 0.0 & 32.0 & 4.5856\\
	\hline
		LOC & When loss of consciousness occurs &Numeric & 0.7428 & 0.0 & 26.0 & 2.6481\\
 	\hline
		SEIZURE & When convulsions are observed & Numeric & 0.1857 & 0.0 & 6.0 & 0.8780\\
	\hline
		BT & Body temperature & Numeric & 37.625 & 35.5 & 40.2 & 1.3041\\
	\hline
		STIFF & Neck stiffness & Numeric & 1.9571 & 0.0 & 5.0 & 1.4033\\
	\hline	
		KERNIG & Kernig sign & Numeric & 0.2142 & 0.0 & 1.0 & 0.4117\\
	\hline	
		LASEGUE & Lasegue sign & Numeric & 0.0785 & 0.0 & 1.0 & 0.2700\\
	\hline	
		GCS & Glasgow coma scale & Numeric & 14.7071 & 9.0 & 15.0 & 1.1536\\
	\hline
		WBC & White blood cell count  & Numeric & 8743.42 & 1070 & 90009 & 7795.80\\
	\hline
		CRP &  C-Reactive protein  & Numeric & 1.6878 & 0.0 & 31.0 & 4.1317\\
	\hline
		ESR & Blood sedimentation test  & Numeric & 5.9285 & 0.0 & 60.0 & 11.880\\
	\hline
		CSF\_CELL & Cell Count in Cerebulospinal Fluid & Numeric & 1505.4 & 0.0 & 63350 & 5708.83\\
	\hline
		Cell\_Poly & Polynuclear cell in CSF  & Numeric & 1025.85 & 0.0 & 61520 & 5402.38\\% \usepackage{array} is required
	\hline
		Cell\_Mono & Mononuclear cell in CSF  & Numeric & 465.08 & 0.0 & 7840 & 816.98\\
	\hline
		CSF\_PRO & Protein in CSF & Numeric & 99.414 & 0.0 & 474.0 & 96.307\\
	\hline
		CSF\_GLU & Glucose in CSF  & Numeric &  56.578& 0.0 & 520 & 44.3412\\
	\hline
		CSF\_CELL3 &  Cell Count CSF 3 days after the treatment & Numeric & 385.18 & 8 & 4860 & 1038.37\\
	\hline
		CSF\_CELL7 & Cell Count of CSF 7 days after treatment& Numeric & 205.61 & 0.0 & 7840 & 816.98\\
	\hline	
	\caption{Numeric Attribute Description.\label{long}}
	\label{table: numeric1}
\end{longtable}

The categorical dataset is given a description to the attribute labels and given a data type. Most of the attributes consist of only 2 values, but does of whom that are multivalued are displayed with the highest and lowest values in the table. See table \ref{table: nominal1} for further insight to the dataset.

 \begin{longtable}{|c|>{\raggedright\arraybackslash}p{4cm}|c|c|c|}
	\hline
	\multicolumn{5}{| c |}{Nominal Attributes}\\
	\hline
	\textbf{Name} & \textbf{Description} &\textbf{Data type} &\textbf{Value 1} &\textbf{Value 2}\\
	\hline
		SEX & Gender of people & Nominal & M (82) & F (58) \\	
	\hline
		Diag2 & Diagnoses & Nominal & VIRUS (98) & BACTERIA (42) \\	
	\hline	
		ONSET & Inception & Nominal &  CHRONIC (1) & ACUTE (130)  \\	
	\hline	
	 	LOC\_DAT & Loss of consciousness& Nominal & - (98) & + (42) \\	
	\hline	
		FOCAL & Focal Sign& Nominal & - (105) & + (35) \\	
	\hline	
		CT\_FIND &  CT findings & Nominal & normal (101) & abnormal (39) \\	
	\hline	
		EEG\_WAVE& Electroencephalography Wave Findings  & Nominal & abnormal (117) & normal (23) \\	
	\hline	
		EEG\_FOCUS & Focal sign in EEG & Nominal & -(104) & +(36) \\	
	\hline	
		CULT\_FIND & If bacteria or virus found & Nominal & F (107) & T(33) \\	
	\hline	
		CULTURE & Name of bacteria/virus found & Nominal & Tb (1) & - (107) \\	
	\hline	
		THERAPY2 & Therapy & Nominal & PIPC+CTX (1) & no\_therapy (58) \\	
	\hline	
		C\_COURSE & Clinical course at discharge & Nominal & negative (117) & paralysis (1) \\	
	\hline	
		COURSE(Grouped) & Grouped attribute of C\_COURSE & Nominal & n (117) & p (23) \\	
	\hline	
		RISK(Grouped) & Class label - at risk  & Nominal & n (121) & p (19) \\	
	\hline		
	\caption{Nominal Attribute Description.\label{long}}
	\label{table: nominal1}		
\end{longtable}

The data above could be divided into a number of sections. Attributes such as AGE and SEX can be categorized as \textit{personal information}. COLD, HEADACHE, NAUSEA LOC etc. can be described as \textit{subject history} as they provide some information on the commencement of the symptoms. BT, STIFF, KERNIG, GCS can be assigned to a category of \textit{physical examination} as they attribute values obtained during investigation. Further more, \textit{laboratory investigation} used to describe the attributes such as WBC, EEG\_WAVE, CULT\_FIND, ESR etc. These are values collected during further investigate of the bodily anomalies. Lastly, \textit{postliminary treatment} if used to describe THERAPY2, CSF\_CELL3 and CSF\_CELL7 as they are attributes describing values after a subject has been treated for meningitis. 


\subsection*{Data Exploration}


\subsection*{Verifying Data Quality}


\prefacesection{Data Preparation}

-  three data preparation techniques to use
- Justify the choices made: Discuss why your chosen techniques are appropriate/required for this data set and mining objective. 
- Document the improvements,

\subsection*{Select Data}
\subsection*{Clean Data}
\subsection*{Construct Data}


\section*{Modeling / Data Mining}

- Use at least two mining algorithms on the dataset. 
- 

\subsection*{Modeling technique}
\subsection*{Test Design}
- Explain how you will evaluate the tests
\subsection*{Build and Assess the Model}

\prefacesection{Evaluation}
- Discuss the overall accuracy of your final model
- non-technical terms, what information you have learned from the dataset.
- Also discuss any limitations of the dataset that may have effected
model accuracy

